% \documentclass[9pt,a5paper]{book}
 \documentclass[9pt,a4paper,oneside]{book}
 \usepackage[margin=0.75in]{geometry}
 \usepackage{times}
 \usepackage{amsmath}  %%% for eg. \text{...}
 \usepackage{graphicx}
 \usepackage{amssymb}  %%% for eg. $\mathbb{R}$
 \usepackage{multirow}
 \usepackage{url}
 \usepackage{longtable}
 \usepackage{hyperref}
 \hypersetup{
  colorlinks   = true,    % Colours links instead of ugly boxes
  urlcolor     = blue,    % Colour for external hyperlinks
  linkcolor    = blue,    % Colour of internal links
  citecolor    = red      % Colour of citations
 }

 \numberwithin{equation}{section}
 \setcounter{secnumdepth}{1}

 %%% Section set-up (each abstract as a section):
% \usepackage{titlesec}
% \titleformat{\section}[display]
%  {\normalfont\huge\bfseries\centering}{\chaptertitlename\ \thechapter}{20pt}{\large}
% \titleformat{\subsection}[display]
%  {\normalfont\huge\centering}{\chaptertitlename\ \thechapter}{20pt}{\normalsize}

 \title{Mathematics of Sea Ice and Ice Sheets  Proceedings}
 \author{Edited by: Susam Boral and Michael Meylan}
 \date{\today}

 \begin{document}
 \maketitle

 \tableofcontents
%
% \clearpage
%
% \part{Front Matter}
%
% \chapter{Introduction}
%
% \chapter{About Mathematics of Sea Ice and Ice Sheets}
%
% \chapter{Venue and Maps}

\newpage

% \part{Abstracts}

%\addcontentsline{toc}{section}{\Large{default category}}
 \section*{Attenuation of long waves through regions of irregular floating ice}
 \label{abs:1}
 \addcontentsline{toc}{subsection}{Attenuation of long waves through regions of irregular floating ice\\ {\bf Richard Porter}} {\bf Richard Porter}, Lloyd Dafydd;\\
 University of Bristol, UK\\
 richard.porter@bris.ac.uk\\


Waves-in-ice measurements are presented, obtained with a shipborne stereoscopic camera system in the Antarctic marginal ice zone during an extreme polar cyclone.
Data show large waves, significant wave height $\approx$5m 45 km in from the ice edge in 100\% sea-ice comprised of pancake floes (60\%) and interstitial frazil ice (40\%), and linear wave decay  ($\approx$0.72\%km$^{-1}$).
The dominant component of wave energy shifts towards longer periods as waves propagate deeper into sea ice, the spectrum becomes narrower in frequency and broader in direction.
Individual waves up to 8m high are observed $\approx$50 km in from the sea-ice edge, the largest wave measured in 100\% sea-ice, but consistent with occurrence probability predicted by linear wave theory.
Measurements reveal that wave-in-ice interactions remain intense in 100\% sea-ice.

 \section*{Where the MIZ really is": Radar altimetry reveals the extent of the wave-affected Antarctic marginal ice zone}
 \label{abs:2}
 \addcontentsline{toc}{subsection}{Where the MIZ really is": Radar altimetry reveals the extent of the wave-affected Antarctic marginal ice zone\\ {\bf Alexander D. Fraser}} {\bf Alexander D. Fraser}, Noah Day, Zhaohui Wang, Luke Bennetts, Siobhan O'Farrell, Richard Coleman, Shiming Xu, Weixin Zhu, Matthis Auger, Lisa Craw, Jill Brouwer, Petra Heil, Chris Horvat;\\
 Institute for Marine and Antarctic Studies (IMAS)
University of Tasmania, Australia\\
adfraser@utas.edu.au\\

The Antarctic marginal ice zone (MIZ) is a crucial region of sea ice where floes break due to wave interaction and ocean-air interaction is enhanced. The extent of this zone has been difficult to detect directly, and a growing body of work shows that the traditional use of "$15$–$80$ sea-ice concentration" does not represent the MIZ well. We have little knowledge of the extent, seasonality or variability of the wave-affected MIZ. Laser altimetry has been used to map this region, but its retrieval suffers from cloud contamination. We show here a new technique to determine the wave-affected MIZ from AltiKa radar altimetry, which is insensitive to cloud cover, thus providing much more coverage. By this technique, MIZ retrieval has been fully automated, resulting in daily, spatially resolved time series of the Antarctic wave-affected MIZ for the first time, from 2013 to the present. The insights from this new dataset can be used to study trends and variability in the MIZ, and also for wave-ice interaction model tuning.

\section*{Motion of Load on an Ice Cover in the Presence of Fluid Layer with Linear Shear Current}
 \label{abs:3}
 \addcontentsline{toc}{subsection}{Motion of Load on an Ice Cover in the Presence of Fluid Layer with Linear Shear Current
\\ {\bf Izolda V. Sturova}} {\bf Izolda V. Sturova}, Larisa A. Tkacheva;\\
Lavrentyev Institute of Hydrodynamics of Siberian Branch of Russian Academy of Science, Novosibirsk, Russia\\
sturova\_i\_v@mail.ru\\

The processes of generation, development, and propagation of spatial flexural-gravity waves are well studied now taking into consideration the case when a fluid in an undisturbed state is at rest or flows at constant speed. However, in real sea conditions, the vertical distribution of current velocity in some cases shows significant changes in magnitude and direction with the depth. This fact indicates that the study of flexural-gravity waves should be also carried out within the framework of such theoretical models that take into account the vertical structure of currents.
 
In the present study, the behavior of an ice cover on the surface of an ideal incompressible fluid of finite depth under the action of a pressure domain that moves rectilinearly at constant velocity in the presence of a linear shear current in the near-surface layer is studied. The fluid flow is not potential. The ice cover is modeled by a thin elastic plate with account for uniform compression. The motion of a load can occur at an arbitrary angle to the direction of current. It is assumed that the ice deflection is steady in the coordinate system moving with the load. The Fourier transform method is used within the framework of the linear wave theory. The critical velocities, the deflection of ice cover and wave forces acting on a moving load are studied depending on the thickness of shear layer, the current velocity gradient, the direction of load motion, and the compression ratio of the ice cover.

\section*{On transitions in water wave propagation through consolidated to broken sea ice covers}
 \label{abs:4}
 \addcontentsline{toc}{subsection}{On transitions in water wave propagation through consolidated to broken sea ice covers
\\ {\bf Luke Bennetts}} {\bf Luke Bennetts}, Jordan Pitt;\\
University of Adelaide, Adelaide, SA 5005, Australia\\
luke.bennetts@adelaide.edu.au\\

A theoretical model is used to study changes in water waves propagating into and through a region containing thin floating ice, for ice covers transitioning from consolidated (large floe sizes) to fully broken (small floe sizes). The degree of breaking is simulated by a mean floe length. It is shown that there are deterministic limits for consolidated and fully broken ice covers, in which the wave fields do not depend on the particular realisation of the floe field for a given mean floe length. The consolidated ice limit is consistent with classic flexural-gravity wave theory, and the fully broken limit is modelled by Bloch waves in a periodic floe field. The transition between the limits shows that the wave field depends on the floe field realization, as multiple wave scattering is a dominant process. The effects of the ice cover on the wave field are quantified using a wavelength, attenuation rate, and a transferred amplitude that represents the amplitude drop at the ice edge. It is shown that as the ice cover breaks up (mean floe size gets smaller), the wavelength decreases, the attenuation rate increases and the amplitude drop decreases (transferred amplitude increases). Dispersion relations are investigated in the consolidated, fully broken and transition regions, and compared to classic models. The results are consistent with physical models and provide a new interpretation of field observations.

\section*{Frequency downshifting in the envelope solitons on the ocean surface covered by ice floes}
 \label{abs:5}
 \addcontentsline{toc}{subsection}{Frequency downshifting in the envelope solitons on the ocean surface covered by ice floes
\\ {\bf Yury A. Stepanyants}} {\bf Yury A. Stepanyants}, Alexey V. Slunyaev;\\
School of Mathematics, Physics and Computing, University of Southern Queensland, Toowoomba, Australia\\
Yury.Stepanyants@unisq.edu.au\\

We present an advanced theoretical study of frequency downshifting in wavetrains propagating on the ocean surface covered by ice floes. Using the empirical model which suggests that small-amplitude surface waves in such an environment decay exponentially with the spatial rate depending on the frequency, ki ~ w n, we derive the downshifting within the wavetrain described by the linear theory. We show that such downshifting appears due to the faster decay of high-frequency components compared to the low-frequency with no energy flux along the spectrum. The alternative weakly nonlinear model is considered within the framework of the nonlinear Schrödinger (NLS) equation augment by the empirical dissipative terms. This model describes the propagation and decay of wavetrains in the form of envelope solitons and accounts for an energy flux along the spectrum down to lower frequencies. For the relatively small dissipation, we derive the frequency downshift in the solitons with the help of the asymptotic approach. The comparison of the downshifting obtained within the framework of linear and nonlinear models shows that in the latter case the frequency downshifting is notably less in comparison with the former case. In conclusion, estimates for the real oceanic conditions will be provided.

\section*{Wave-induced flexure on multiple Antarctic ice shelves}
 \label{abs:6}
 \addcontentsline{toc}{subsection}{Wave-induced flexure on multiple Antarctic ice shelves
\\ {\bf Jie Liang}} {\bf Jie Liang}, Luke G. Bennets, Jordan Pitt;\\
University of Adelaide, Adelaide, SA 5005, Australia\\
jie.liang02@adelaide.edu.au\\

There has been rising interest in ice shelf flexure produced by ocean waves due to a combination of recent theoretical studies and observed calving events. Current mathematical models that explore the flexure of ice shelves induced by ocean waves often employ a simplified uniform geometry, consisting of constant shelf thickness and seabed. Bennetts et al. (2022) introduced an efficient mathematical model of the wave-induced shelf flexure for the non-uniform geometry of the Ross Ice Shelf taken from the BEDMAP2 dataset. The results showed significant impact of shelf geometry and wave period on flexural strain. 
The presented study further investigates the impact of geometrical variations on shelf flexure. It extends the work of Bennetts et al. (2022) to multiple ice shelves, providing statistical relationships between shelf flexure and ice shelf thickness and seabed variation across the range of wave periods (swell, infra-gravity, and extremely long period waves).\\
\textbf{References:}
Bennetts, L. G., Liang, J., and Pitt, J. (2022). Modelling ocean wave transfer to ross ice shelf flexure. Geophysical Research Letters, 49(21): e2022GL100868.

\section*{Analyzing Wave Attenuation in the Antarctic Sea Ice Using Ship Motion-Derived Wave Spectra}
 \label{abs:7}
 \addcontentsline{toc}{subsection}{Analyzing Wave Attenuation in the Antarctic Sea Ice Using Ship Motion-Derived Wave Spectra
\\ {\bf Anran Duan }} {\bf Anran Duan};\\
University of Melbourne, Australia\\
anrand@student.unimelb.edu.au\\

Studying the interaction between ocean waves and sea ice cover is essential for ensuring safety operations within the polar region and investigating the effects of climate change. However, while spectral wave modelling in open ocean areas has achieved high accuracy, modelling in the marginal ice zone remains challenging and is associated with relatively large errors. This challenge primarily stems from the limited availability of observations of waves in sea ice. 
In the present study, the wave-buoy analogy approach is applied to the extensive ship motion datasets collected during the cruises in the Southern Ocean to produce wave spectra estimates. By applying the transfer function called the response amplitude operator (RAO), the ship response spectrum can be translated into wave energy spectrum. The RAO is computed utilizing the ship hydrodynamic model NEMOH. The input includes the hull geometry with an estimated draft, as the actual draft is unavailable. The RAO is then calibrated by adjusting the draft and comparing the significant wave height of the measured wave spectra with the satellite altimeter observations, relying on the altimeter’s accuracy over open ocean, until the difference between them is minimised. The sea state parameters from the corrected spectra show good agreement with those from the ERA5 reanalysis dataset. 
In the next step, wave attenuation is analyzed based on this database of wave spectra. Results show a power-law relationship between attenuation rate and frequency, which has been reported in several previous studies, and the power value we obtained is comparable to previous findings.


\section*{Predicting coastal polynyas in Antarctica from regional topography and large-scale atmospheric circulations
}
 \label{abs:8}
 \addcontentsline{toc}{subsection}{Predicting coastal polynyas in Antarctica from regional topography and large-scale atmospheric circulations
\\ {\bf Ihanshu N. Rane}} {\bf Ihanshu N. Rane}, Stefan Jendersie;\\
Antarctic Research Centre, Victoria University of Wellington, New Zealand\\
ihanshu.rane@vuw.ac.nz\\

Coastal polynyas are areas of open water enclosed by winter sea ice in Antarctica.
Polynyas significantly impact atmospheric, cryospheric and oceanic processes on different 
spatial and temporal scales. Very cold, local high-speed winds drive the sea-ice away from 
the coast, while at the same time intensifying the generation of new sea ice. The associated
Brine rejection leads to formation of very dense shelf water. This dense water mass 
eventually drives the global thermohaline circulation and ventilates the world’s deep ocean 
benthic ecosystems.
For studying future climate, it is important to include the formation and the intense 
sea-ice production of coastal polynyas in global climate models. Currently, they are not 
represented because the various polynya-associated processes act on length scales below the 
resolution limit of global simulations.
This project addresses two research questions using machine learning algorithms.
First, how does the shape of local coastal topography promote or inhibit the occurrence of
coastal polynyas. Second, how large-scale synoptic processes trigger and enhance local high-speed winds under the influence of regional orography. In a two-stage approach, we first train 
a Random Forest (RF) algorithm using surface winds, coastline shape and sea-ice freeboard 
as input variables. Then, large-scale pressure and gradient wind variables will be added for 
training. The expected output from the RF-Classifier represents the binary prediction of 
polynya occurrence, the RF-Regressor predicts the area of polynyas. The overall objective is 
to predict polynyas using regional topography and large-scale atmospheric circulation only, 
thus avoiding the need to resolve local processes in climate models

\section*{Nonlinear gravity waves in a channel covered by broken ice}
 \label{abs:9}
 \addcontentsline{toc}{subsection}{Nonlinearity in hydroelastic interaction
\\ {\bf Yuriy Semenov}}  {\bf Yuriy Semenov} and Bao-yu Ni;\\
College of Shipbuilding Engineering, Harbin Engineering University, Harbin, China \\
yuriy.a.semenov@outlook.com\\

The two-dimensional nonlinear problem of a steady flow in a channel covered by broken ice with an arbitrary bottom topography including a semi-circular obstruction is considered. The mathematical model is based on the velocity potential theory accounting for nonlinear boundary conditions on the bottom of the channel and at the interface between the liquid and the layer of the broken ice, which are coupled through a numerical procedure. A mass loading model together with a viscous layer model is used to model the ice cover. The integral hodograph method is employed to derive the complex velocity potential of the flow, which contains the velocity magnitude at the interface in explicit form. The coupled problem is reduced to a system of integral equations in the unknown velocity magnitude at the interface, which is solved numerically using a collocation method. Case studies are conducted both for the subcritical and for the supercritical flow regimes in the channel. For subcritical flows, it is found that the ice cover allows for generating waves with amplitudes larger than those that may exist in the free surface case; the ice cover prevents the formation of a cusp and extends the solution to larger obstruction heights on the bottom. For supercritical flow regimes, the broken ice significantly affects the waveform of the soliton wave making it gentler. The viscosity factor of the model apparently governs the wave attenuation. 

\section*{Hydroelastic waves in a frozen channel with compressed floating ice}
 \label{abs:10}
 \addcontentsline{toc}{subsection}{Hydroelastic waves in a frozen channel with compressed floating ice
\\ {\bf Tatyana Khabakhpasheva}}  {\bf Tatyana Khabakhpasheva} and Evgeniy Batyaev ;\\
 University of East Anglia, UK;\\
Lavrentyev Institute of Hydrodynamics SB RAS, Novosibirsk, Russia\\
tana@hydro.nsc.ru\\

Hydroelastic wave propagation in an ice-covered channel for the compressed ice cover has been investigated. The channel has a rectangular cross-section. The fluid in the channel is inviscid, incompressible and covered with ice. The ice is modeled as a thin elastic plate. Cases of ice compression along the channel and across the channel are considered.

Methods for obtaining dispersion relations and determining critical velocities with the help of normal modes for uniform non-compressed plates have been developed. The problem is reduced to the problem of the wave profiles across the channel. The solution to the considered problem is obtained in the form of a series of the vibration modes of an elastic beam without compression; see [1-3]. It is shown that phase velocities and values of the frequencies of propagating waves stay low with increasing compression. Abnormal behaviour was observed for the first mode of “very compressed” ice if the compression was along the channel.  For compression across the channel, the existence of two modes with the same frequency was found. It was obtained that the second mode wave can propagate faster than the first mode wave in the case of “very compressed” ice.
\\
\textbf{References}
\begin{enumerate}
    \item A. A. Korobkin, T. I. Khabakhpasheva, A. A. Papin, Waves propagating along a channel with ice cover European Journal of Mechanics - B/Fluids. 2014. V.47. pp.166–175.
    \item E. A. Batyaev, T.I. Khabakhpasheva, Hydroelastic waves in a channel covered with a free ice sheet  Fluid Dynamics. V.50 (6). pp.775-788.
    \item K. Shishmarev,  K. Zavyalova,  Ev. Batyaev, T. Khabakhpasheva. Hydroelastic Waves in a Frozen Channel with Non-Uniform Thickness of Ice. Water. 2022; 14 (3):281.
\end{enumerate}

\section*{Characteristics of periodic hydroelastic waves in a frozen channel with a non-uniform thickness of ice.}
 \label{abs:11}
 \addcontentsline{toc}{subsection}{Characteristics of periodic hydroelastic waves in a frozen channel with a nonuniform thickness of ice.
\\ {\bf Konstantin Shishmarev}} {\bf Konstantin Shishmarev}, K.N. Oglezneva and T.I. Khabakhpasheva;\\
College of Shipbuilding Engineering, Harbin Engineering University, Harbin, China\\
shishmarev.k@mail.ru\\

Linear hydroelastic waves propagating in an ice channel are investigated. The ice thickness is constant along the channel and variable across the channel. The channel is a rectangular cross-section with finite depth and infinite extent. The liquid in the channel is inviscid and incompressible. The liquid flow caused by the ice deflection has potential. The ice is modeled by a thin elastic plate clamped to the channel walls. The coupled hydroelastic problem is reduced to the problem of the wave profiles across the channel. The wave profiles are sought as a series of normal dry modes of the plate, the coefficients of which are to be determined. The dispersion relations of these hydroelastic waves, their phase speeds, and their profiles across the channel are determined.

Two cases of variation in the ice thickness are considered. The first case is symmetric linear change with respect to the centre of the channel. The second case is an arbitrary shape of the ice thickness. In both cases, the method of construction of the normal dry modes are presented. In the first case the solution is obtained in the form of a series of Bessel functions using symmetry of the wave profiles. The normal modes of the plate with an arbitrary thickness are calculated using piecewise linear approximation of the plate thickness. These modes are such that the corresponding deflections, slopes, bending stresses and shear forces are continuous. The modes are presented by Bessel functions in the intervals, where the plate thickness is linear, and by trigonometric and hyperbolic functions, where the thickness is approximated as constant.

\section*{Modelling Wave-Ice Interactions in Three-Dimensions in the Marginal Ice Zone of the Beaufort Sea}
 \label{abs:12}
 \addcontentsline{toc}{subsection}{Modelling Wave-Ice Interactions in Three-Dimensions in the Marginal Ice Zone of the Beaufort Sea
\\ {\bf Will Perrie}} {\bf Will Perrie}, Michael H. Meylan, Bechara Toulany, Michael P. Casey and Yongcun Hu;\\
Bedford Institute of Oceanography, Canada\\
William.Perrie@dfo-mpo.gc.ca\\

This study is about wave-ice interactions in the marginal ice zone (MIZ). We compare simulations using recent three-dimensional formulations for wave-ice interactions and the scattering of ocean surface waves by flexible ice floes, with older parameterizations and field observations. These parameterizations are implemented in a modern version of the wave model WAVEWATCH III®, as source terms in the action balance equation. Field observations consist of wave characteristics collected during the Arctic Sea State Boundary Layer Experiment of 2015 in the Beaufort Sea. Comparisons focus wave heights, 1-dimensional and 2-dimensional wave spectra and wave attenuation as the waves propagate into the MIZ. Results show that the new wave-ice formulations for wave-ice interactions provide an improvement compared to previous parameterizations, in the simulations of observation cases. Test cases include collinear winds and waves, and crosswinds to the waves, propagating into the MIZ.

\section*{Surface wave attenuation in seasonal sea ice observed in high resolution with seafloor telecommunication cables}
 \label{abs:13}
 \addcontentsline{toc}{subsection}{Surface wave attenuation in seasonal sea ice observed in high resolution with seafloor telecommunication cables
\\ {\bf Maddie Smith}} {\bf Maddie Smith}, Jim Thomson, Michael Baker, Rob Abbott and Jake Davis;\\
Woods Hole Oceanographic Institution\\
madisonmsmith@whoi.edu\\

The attenuation of ocean surface waves in seasonal sea ice and resulting sea ice motion is challenging to observe with conventional methods using sparse mooring or buoy arrays. A novel method for persistent observation of wave-ice interactions using distributed acoustic sensing (DAS) along seafloor fiber optic telecommunication cables is applied to 36-km cross-shore cable in the Alaskan Arctic on the Beaufort Shelf. DAS optical sensing of fiber strain-rate provides a proxy for seafloor pressure, which we calibrate with wave buoy measurements during the ice-free season (August 2022). This calibration is applied to the ice formation season (November 2021) to obtain unprecedented resolution of variable wave attenuation rates in new, partial ice cover. The rapid evolution of the location and strength of attenuation serves as proxy for the evolution of ice coverage and thickness, especially during rapidly evolving events. Results suggest that higher attenuation rates previously observed near the ice-edge may be a result of wave-ice interactions leading to ice compaction and increased thickness. Such high-resolution estimates of wave attenuation will contribute to better understanding the range of wave attenuation coefficients appropriate for different ice types and thicknesses, and implementation in coupled wave-sea ice models. 

\section*{Vertical motion of a circular cylinder in water covered by an ice sheet}
 \label{abs:14}
 \addcontentsline{toc}{subsection}{Vertical motion of a circular cylinder in water covered by an ice sheet
\\ {\bf Hang Xiong}} {\bf Hang Xiong}, Bao-Yu Ni, Alexander Korobkin;\\
College of Shipbuilding Engineering, Harbin Engineering University, Harbin, China\\
xionghang@hrbeu.edu.cn\\

The two-dimensional unsteady problem of a rigid cylinder moving vertically in a liquid covered by an elastic thin plate of infinite extent is studied. The liquid is incompressible and inviscid. The flow caused by the body motion is potential. Deflections of the elastic plate are assumed small compared with the radius of the
cylinder. The boundary condition on the plate/liquid interface is linearised and imposed on the initial equilibrium level of the plate. It is shown that the problem is coupled for impulsive motions of the body and decoupled if the body acceleration is moderate. The coupled problem is solved using the conformal mapping of the flow region onto a ring and the method of separation of variables applied to both the hydrodynamic and structural parts of the problem. This study is restricted to the constant speed of the cylinder motion towards the ice plate. The effect of the nonlinear term in the hydrodynamic pressure acting on the plate/liquid interface on the response of the ice plate is also investigated.

\section*{Long-term measurements of waves and sea ice in the Beaufort Sea}
 \label{abs:15}
 \addcontentsline{toc}{subsection}{Long-term measurements of waves and sea ice in the Beaufort Sea
\\ {\bf J. Thomson}} {\bf J. Thomson}, M.L. Timmermans (Yale), J. O'Brien (WHOI), and I. Le Bras (WHOI);\\
Applied Physics Lab, University of Washington, USA\\
jthomson@apl.washington.edu\\

We present an update on an ongoing effort to measure ocean surface waves and sea
ice in the central Beaufort Sea of the western Arctic. The time series began in 2012 and
now has twelve years of hourly wave spectra and ice draft estimates at two mooring
locations within the Beaufort Gyre Observing System (BGOS). The data products are
publicly available and ready for testing coupled wave-ice models. To date, the data
have been used to investigate the fetch dependence of waves in the western Arctic
(Thomson and Rogers, 2014), the local generation of waves in partial ice cover (Cooper
et al, 2022), and large-scale budgets for wave-driven energy and momentum fluxes
(Thomson, 2022). We review these results and explore new applications for large
wave-ice datasets.

\section*{Triad resonances of flexural gravity wave with shear current}
 \label{abs:16}
 \addcontentsline{toc}{subsection}{Triad resonances of flexural gravity wave with shear current
\\ {\bf N. Bisht}} {\bf Neha Bisht}, S. Boral and T. Sahoo;\\
Dept. of Ocean Engineering and Naval Architecture, Indian Institute of Technology Kharagpur, Kharagpur, India\\
bishti.neha@gmail.com\\

Recent decades have seen a rise in the investigation of surface gravity waves with very large floating structures. An analogous branch of study is wave-ice interaction in cold region science and technology. In both the class of problems, the floating structures are assumed to be thin, homogenous and isotropic flexible plates. The hydroelastic responses of these structures are explored by analyzing flexural gravity waves resulting from the interaction of surface gravity waves with the large floating structures. Furthermore, ocean currents, which occur in virtually every region of the ocean, contribute to the complexity of the issues. Recent research on wave-structure interaction problems has revealed the phenomenon of wave blocking, which occurs due to the interaction between flexural gravity waves and ocean current with/without the existence of lateral compressive stress. In this context, the formation of triads of flexural gravity waves has gained momentum as it entails the distribution of wave energy on the ice-covered ocean surface. This study manifests the formation of triads in flexural gravity waves in the presence of uniform ocean currents with constant linear vorticity and structural compressive force within the framework of wave blocking. Triad formations are explored geometrically using Ball's diagram for different compressive force and current speed values. The study assumes small amplitude water wave theory and homogeneous fluid-structure response for finite water depth. Besides, the study reveals that at least three triads occur for certain frequencies within the blocking limits, depending on the speed of current and the magnitude of lateral compressive stress.

\section*{Flexural gravity wave interaction with a circular polynya in the context of wave blocking}
 \label{abs:17}
 \addcontentsline{toc}{subsection}{Flexural gravity wave interaction with a circular polynya in the context of wave blocking
\\ {\bf P. Negi}} {\bf P. Negi}, T. Sahoo and M.H. Meylan;\\
Dept. of Ocean Engineering and Naval Architecture, Indian Institute of Technology Kharagpur, Kharagpur, India\\
pavan.negi04@gmail.com\\

The study of ocean wave propagation through floating ice sheets has gained significant attention
owing to its relevance in the polar region’s science and technology. Further, various geophysical
processes, viz., upwelling, wind stresses and geothermal effects, could have a notable impact on ice
sheets in polar regions. Besides, there are open water unfrozen areas within the ice-covered ocean
known as polynyas, which are created in areas where the ice motion is divergent because of the
prevailing winds or oceanic currents. It is interesting to know the interaction of flexural gravity waves
outside these polynyas with the surface gravity waves within these polynyas. Of particular interest
is the phenomenon of wave blocking, a distinctive characteristic associated with the propagation of
flexural gravity waves. This phenomenon unveils the presence of multiple propagating wave modes
within specific periods under the influence of lateral compressive forces. Recent studies on flexural
gravity waves have unveiled that wave blocking occurs when the group velocity reaches zero while
the phase velocity remains positive for distinct periods. In the present analysis, the scattering of
flexural gravity waves by a circular polynya within the regime of wave blocking is studied in finite
water depth. The Fourier-Bessel series expansion is employed to determine the velocity potential
for single and multiple propagating wave modes within the ice sheet. The study further investigates
the influence of various hydrodynamic and structural parameters on the scattered wave field, wave
elevation within the polynya, and the strain rate experienced by the ice sheet.

\section*{Spectral Analysis of Ice Shelf Vibrations}
 \label{abs:18}
 \addcontentsline{toc}{subsection}{Spectral Analysis of Ice Shelf Vibrations
\\ {\bf R. Aljabri}} {\bf R. Aljabri},  and M.H. Meylan;\\
School of information and physical sciences, University of Newcastle, Australia\\
Rehab.Aljabri@uon.edu.au\\

In this work, our goal is to develop a method that calculates the vibrations of an ice shelf floating on shallow water under different boundary conditions. Furthermore, we aim to extend the method to find the mode shapes of the ice shelf-water system. These mode shapes describe the behavior of the system and give us a clear explanation of what happened in the ice shelf/cavity system. Moreover, we present motion simulations of the system for the potential velocity of the water and the vertical displacement of the plate at various times under different boundary conditions.

\section*{A simulative study on a breakable large scale ice floe in the presence of current}
 \label{abs:19}
 \addcontentsline{toc}{subsection}{A simulative study on a breakable large scale ice floe in the presence of current
\\ {\bf L. Sun}} {\bf L. Sun}, B.-Y. Ni and S. Boral;\\
College of Shipbuilding Engineering, Harbin Engineering University, Harbin, China\\
3084153944@qq.com\\

In the past few decades, a significant progress is observed in the study on surface gravity wave interaction with large floating ice sheets or a large-scale ice floes stripped from the ice sheet in the marginal ice zone (MIZ) due to rise in global temperature as well as global warming. In most of the studies, large floating ice sheets are considered as an infinitely extended thin floating elastic plate, however, due to strong wind, high waves and temperature variation the ice sheets in the marginal ice zone behaves like fractured ice. Furthermore, the study on wave-ice interaction problems becomes more complex with the inclusion of the role of ocean current. Since, the dispersion curve loses its symmetricity about the axes and as a result for each frequency two-distinct wavenumbers exists. 

The present study deals with the surface gravity wave propagation in the presence of current in which the ice sheet is considered as fractured ice under the assumption of linearized theory of water waves and small amplitude structural response.  Emphasis is given to investigate numerically the floe size distribution of a large scale ice floe in the presence of current in a water of infinite depth. The simulative study is based on CFD-DEM method in which volume of fluid (VOF) method is used to simulate the water phase and discrete element method (DEM) is considered to model the large scale floating ice. In addition, a probability distribution, namely, Weibull distribution is considered for analyzing floe size distribution (FSD). The impact of wave-current field on a large-scale floating ice in established based on a two and three dimensional wave numerical model. The study reveals that the fracture length changes with a change in the current, which is due to the change in wavelength for the same frequency. Additionally, the presence of current causes the drift motion of the ice sheets. Moreover, it is observed that the size distribution of fractured ice is uniform in the case of following current, whilst the same becomes concentrated in a small scale range in the case of opposing current. Additionally, the averaged normalized size of the broken ice increases/decreases in the presence of following/opposing current. This is because of the fact that wavelength increases (decreases) as the wave propagates in the same (opposite) direction of current.

\section*{The impact of a flexible bed on the interaction between oblique waves and multiple surface-piercing porous barriers}
 \label{abs:20}
 \addcontentsline{toc}{subsection}{The impact of a flexible bed on the interaction between oblique waves and multiple surface-piercing porous barriers
\\ {\bf B. Sarkar}} {\bf B. Sarkar}, and S. De;\\
University of Calcutta, India\\
biman228sarkar@gmail.com\\

Utilizing the linearized theory of water waves, we examine a model that delves into the scattering of oblique waves by obstacles. The obstacles, in this case, take the form of thin multiple surface-piercing porous barriers characterized by non-uniform porosity. The study is conducted within the context of a flexible ocean bed of uniform finite depth. The flexible base surface is represented as a thin elastic plate, relating to the Euler–Bernoulli beam equation. Employing the eigenfunction expansion method and mode-coupling relations, we derive four Fredholm-type integral equations from the boundary value problem. To solve these integral equations, multi-term Galerkin approximations are applied using Chebyshev polynomials multiplied by appropriate weight functions. The obtained analytic solutions reveal various hydrodynamic quantities such as reflection coefficients, transmission coefficients, dissipated wave energy, and non-dimensional wave force. These quantities are graphically presented for different values of dimensionless parameters. Notably, the graphical representations highlight the significant influence of barrier permeability and bottom surface thickness in modelling effective breakwaters.

\section*{Mitigation of hydrodynamics response of water wave on an Elastic plate in the presence of a Bottom-Standing Porous Structure}
 \label{abs:21}
 \addcontentsline{toc}{subsection}{Mitigation of hydrodynamics response of water wave on an Elastic plate in the presence of a Bottom-Standing Porous Structure
\\ {\bf G. Sahoo}} {\bf G. Sahoo}, S. Singla and S. C. Martha;\\
Department of Mathematics, Indian Institute of Technology Ropar, Rupnagar-140001, India\\
gagan.19maz0002@iitrpr.ac.in\\

This study explores the interaction of oblique water waves with a finite-width bottom-standing porous structure, taking into account the presence of an elastic plate within the framework of linearized water wave theory. The wave passed the thick porous structure employs the Sollitt and Cross model, while the elastic plate is modelled using the thin plate theory. Utilizing the eigenfunction expansion method, the corresponding boundary value
problem is transformed into a set of linear algebraic equations. The system is then solved numerically, and important parameters such as reflection, transmission, dissipation coefficients, shear force, and strain are computed and presented graphically. It has been found that as the bottom-standing porous structure’s length, width, porosity, and friction factor grows, the transmission coefficient decreases and the dissipation coefficient rises, which helps in diminishing the wave impact on the elastic plate.

\section*{Flexural gravity wave scattering due to seabed undulation in the presence of multiple propagating wave modes}
 \label{abs:22}
 \addcontentsline{toc}{subsection}{Flexural gravity wave scattering due to seabed undulation in the presence of multiple propagating wave modes
\\ {\bf T. Sahoo}} {\bf T. Sahoo}, S. Boral, M. H. Meylan and B.-Y. Ni;\\
Department of Ocean Engineering and Naval Architecture, IIT Kharagpur, India\\
tsahoo@naval.iitkgp.ac.in\\

Flexural gravity waves are generated due to the interaction of surface gravity waves with a floating flexible plate. These waves are the result of the interplay between gravitational forces and elastic bending, making them distinct from conventional surface gravity waves. In coastal environments as well as in polar regions, the interactions of flexural gravity waves with bottom steps and submerged breakwaters are of significant importance. The recent study on blocking dynamics of flexural gravity waves has revealed the existence of multiple propagating wave modes for certain frequency and lateral compressive force, whilst major studies on flexural gravity wave transformation are limited to single propagating wave. Besides, the higher-order boundary condition on the ice-covered ocean surface in wave-ice interaction problems add additional challenges since the vertical eigenfunctions involving flexural gravity waves are not orthogonal in the classical sense. In this presentation, flexural gravity wave scattering due to step type undulated sea bed is studied in the case of multiple propagating waves. The method is extended to the case of a submerged rectangular breakwater and trenches using geometrical symmetry. The orthogonal eigenfunctions associated with surface gravity waves are used to develop a system of equations in the eigenfunction expansion method involving the flexural gravity wave problem, which is found to be much easier than the use of the orthogonal mode-coupling relation associated with flexural gravity waves. The energy balance relation is obtained to account for multiple propagating wave mode and is used to validate the computational result. Various numerical results will be presented to demonstrate the effect of multiple propagating waves in the context of wave blocking.

\end{document}